\documentclass[xcolor=table]{article}
\usepackage[utf8]{inputenc}

% documentos multi columna
\usepackage{multicol}         

% colores en el documento
\usepackage[table,xcdraw]{xcolor}

\usepackage{soul}

% "American Mathematics Society"
\usepackage{amsmath}   

\usepackage[makeroom]{cancel}

% símbolos de "American Mathematics Society"
\usepackage{amssymb}                    

% control extenso de headers & footers
\usepackage{fancyhdr}                   

% controlar imágenes
\usepackage[pdftex]{graphicx}           

% esquemáticos estandarizados
\usepackage[american, oldvoltagedirection]{circuitikz}

% extiende el paquete de graphicx
\usepackage{adjustbox}

% posicionamiento de imágenes
\usepackage{float}

% tipografías de distintos idiomas
\usepackage[spanish, es-tabla]{babel}

% imágenes embebidas
\usepackage{wrapfig}

% customizacion de paquetes
\usepackage{caption}

% definición de márgenes y tamaño de papel
\usepackage[a4paper, lmargin=10mm, rmargin=10mm, tmargin=17mm, bmargin=30mm]{geometry}

% tabulares de múltiples filas
\usepackage{multirow}

% variantes de fuentes (usado para escribir indicadoras)
\usepackage{bbm}

% entornos gráficos encapsulados (para enunciados)
\usepackage{framed}


% extensión para paquete framed
\usepackage{mdframed}

% grilla para poder aumentar la precisión de la edición gráfica
%\usepackage[grid, gridcolor=red!40, subgridcolor=green!40, gridunit=cm]{eso-pic}

% flechas para referencias en circuitikz
\usetikzlibrary{arrows}

\usetikzlibrary{shapes.geometric, positioning, arrows.meta}

% texto de ejemplo
\usepackage{lipsum}

% paquete para listar código
\usepackage{listings}

\usepackage{hyperref}

\hypersetup{
    colorlinks=true,
    linkcolor=black,
    filecolor=magenta,      
    urlcolor=blue,
}

\urlstyle{same}

\usepackage{tikz}

\usepackage{dirtytalk}

\usepackage{subcaption}

\usepackage{mwe}



\usepackage[toc,page]{appendix}
