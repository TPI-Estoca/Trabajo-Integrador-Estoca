\newcommand{\hatv}[1]{\overset{\wedge}{\underset{\vee}{\mathstrut#1}}}
\newcommand{\fhatv}{\hatv{f}}
\newcommand{\Uhatv}{\hatv{U}}
\newcommand{\Tau}{\mathcal{T}}
\newcommand{\indice}{
\tableofcontents
\newpage
}
\newcommand{\cover}{\input{00_modules/03a_cover}}
\newcommand{\redcover}{\input{00_modules/03b_redcover}}

% highlight en rojo
\sethlcolor{red}
%Para tener subsections con letras
%\renewcommand{\thesubsection}{\thesection.\alph{subsection}}

%Para tener subsubsections con letras
\renewcommand{\thesubsubsection}{\thesubsection.\alph{subsubsection}}

%Shortcut para matrices (Control I me hizo necesitar esto)
\newcommand\m[1]{\begin{bmatrix}#1\end{bmatrix}}

%Shortcut para ecuaciones numeradas
\newcommand\eqn[1]{\begin{equation}#1\end{equation}}

%Shortcut para ecuaciones no numeradas
\newcommand\eqnn[1]{\begin{equation*}#1\end{equation*}}

\newcommand{\norm}[1]{\left\lVert#1\right\rVert}

\newcommand\mdf[1]{\begin{mdframed}[backgroundcolor=blue!12]#1\end{mdframed}}


\newcommand\bm[1]{\begin{bmatrix}#1\end{bmatrix}}

\newcommand*\circled[1]{\tikz[baseline=(char.base)]{
            \node[shape=circle,draw,inner sep=2pt] (char) {#1};}}

% Barra larga bajo las paginas
\renewcommand{\footrulewidth}{0.5pt}

\renewcommand{\theenumi}{\alph{enumi}}

\usepackage{enumitem}

\pagestyle{fancy}
\fancyhf{}
\setlength{\headheight}{40pt}

\rhead{\includegraphics[scale=0.15]{pics/carátula/fiuba_logo.jpg}}

\lhead{\footnotesize 
    TP1 Redes de Comunicaciones- 1°C 2025
}   
    
\rfoot{\thepage}
\lfoot{Cavalitto, Emilia}